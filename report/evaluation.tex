\subsection{Correctness}
The souffle pretty-printer is used to output a Datalog program that may be evaluated by souffle. The process of comparing the output of souffle to that of the internal interpreter has been automated and a range of tests written. If the tests agree then the souffle pretty-printer is said to be correct for the test program. Given that souffle correctly implements Datalog, the interpreter too is concluded to be correct for the given test program. The test cases are selected to cover mutual recursion, negation, meta-predicates, and the other various language extensions.

\subsection{Performance}
Souffle implements semi-naive evaluation\cite{Green:2013:DRQ:2688167.2688168} which is the same as naive evaluation except that it utilizes the following key-insight. An instantiation of the terms of a rule may derive new tuple(s) if and only if at least one tuple that was derived in the previous iteration is used in the instantiation.
The Nat example with an upper limit of $N$ highlights the difference in performance that this may give.
\begin{minted}{yaml}
Nat(0).
Nat(y) :- Nat(x), BIND(y, x + 1), y <= N.
\end{minted}
\noindent
To evaluate using naive evaluation, in each step with $k$ unary tuples in $Nat$, the evaluation must Select all tuples from Nat, calculate y for each x, filter all calculated y with the less than rule, project the $y$ component into the new Nat relation, and take the union between the old and the new Nat relation. With the current implementation, union has time complexity $O(k \cdot log(k))$ and the remaining operations are $O(k)$. The expected time-complexity for the NAT-example is thus in the order of:
\begin{align*}
\sum_{k = 1}^{N} 4 k + k \cdot log (k) \leq 4 N^2 + N^2 log(N) = O(N^2 log(N))
\end{align*}
For semi-naive evaluation, each iteration gives a single new element to consider. The corresponding time complexity thus reduces to the order of:
\begin{align*}
	\sum_{k = 1}^{N} 4 + log (k) \leq 4N + N log(N) = O(N log(N))
\end{align*}

\subsubsection{Experimental Results}



\subsection{Expressive Power}
\textit{To be Written ... }