\section{Appendix A: Abstract Algebra for Rule Evaluation}
Consider the following example:
\begin{align*}
&Order(1, 2),\; Order(2, 3). \quad\quad\quad\quad\quad\quad\quad\quad\quad[r_1]\\
&Order(x, z) \coloneqtwo Order(x, y),\; Order(y, z). \;\quad\quad\quad[r_2]
\end{align*}
\noindent
Rule $[r_1]$ states that the orders $1, 2$ and $2, 3$ holds. Rule $[r_2]$ states that the binary $Order$ relation is transitive. The BUN evaluation proceeds as follows. For each rule that is evaluated, an artificial body-relation $B$ is introduced. The body-relation will be incrementally populated and extended through the evaluation of the atoms in the body. 

Initially, the relation associated with each atom is unnamed, i.e. the columns of the relation have no name-restrictions. Given a rule $r$, we order the atoms from the body of the rule as $A_1, \ldots, A_n$. We will consider each atom in turn and use the body relation $B$ as an accumulator. The equations describing the rule-evaluation process is as follows (the notation is explained below):
\begin{align*}
B^{0} &\leftarrow \top \\
B^{i + 1} &\leftarrow B^{i} \bowtie \sigmatwo_{Term(A_{i + 1})}\;(A_{i + 1}), \; i = 0, \ldots, n - 1\\
H & \leftarrow H \cup \Pi_{Term(H)}(B^n)
\end{align*}
\noindent
The body relation is initially assigned to $\top$ denoting an unknown relation: $\top \bowtie R = R \bowtie \top = R$. The $Term$ function gives the terms in the given atom. A selection is then done for the atom. The special selection operator is denoted $\sigmatwo$. Informally it selects a set of tuples from the relation associated with the atom that satisfies the constraints imposed by the terms of the atom. As a more formal example:
\begin{align*}
\sigmatwo_{x,x,y,c} = \rho_{x/x_1, y/y_1} \circ \Pi_{x_1, y_1} \circ \sigma_{x_1 = x_2, C_1 = c} \circ \rho_{x_1/N_0, x_2/N_1, y_1/N_2, C_1/N_3}
\end{align*}

The selection is for three variables $x, x, y$ and a constant $c$. First the rename operator $\rho$ is used to rename the columns of the relation ($N_i$ is an artificial initial name for column $i$). Then the ordinary $\sigma$ operator selects all tuples such that the corresponding variables and constants match under the given naming. The result of the selection is then projected (discarding duplicates and constants) using the projection operator $\Pi$, and finally the inverse renaming is performed.

The selection result is joined ($\bowtie$) with the current body relation $B^{i}$ to form the next body relation $B^{i + 1}$. In our example (assuming that $r_1$ has been evaluated) we get:
\begin{align*}
B^{1} &\leftarrow \sigmatwo_{x, y}\;(Order) = \{(1, 2), (2, 3)\}_{x, y}\\
B^{2} &\leftarrow B^{1} \bowtie (\sigmatwo_{y, z}\;(Order) = \{(1, 2), (2, 3)\}_{y, z})\\
      &= \{(1, 2), (2, 3)\}_{x, y} \bowtie \{(1, 2), (2, 3)\}_{y, z}\\
      &= \{(1,2,3)\}_{x,y,z}
\end{align*}
\noindent
Finally we project the result of $B^n$ and add the new tuples to the head relation:
\begin{align*}
Order & \leftarrow Order \cup (\Pi_{x,z}(\{(1,2,3)\}_{x,y,z}) = \{(1,3)\})
\end{align*}
\noindent
The process is iterated until a fix-point is found, i.e. until no new tuples can be derived from the set of rules. In our example, iterating $[r_2]$ again gives no new tuples and so the $Order$ relation has been computed as: $\{(1,2), (1,3), (2,3) \}$.