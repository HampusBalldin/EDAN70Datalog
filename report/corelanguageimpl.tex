The current query evaluation mechanism is a bottom-up naive (BUN) \cite{Green:2013:DRQ:2688167.2688168} evaluation. It is based on the fixpoint-theoretic semantics. The implementation is best introduced through an example.
\begin{align*}
&Order(1, 2),\; Order(2, 3). \quad\quad\quad\quad\quad\quad\quad\quad\quad[r_1]\\
&Order(x, z) \coloneqtwo Order(x, y),\; Order(y, z). \;\quad\quad\quad[r_2]
\end{align*}
\noindent
Rule $[r_1]$ states that the orders $1, 2$ and $2, 3$ holds. Rule $[r_2]$ states that the binary $Order$ relation is transitive. The BUN evaluation proceeds as follows. For each rule that is evaluated, an artificial body-relation $B$ is introduced. The body-relation will be incrementally populated and extended through the evaluation of the atoms in the body. 

Initially, the relation associated with each atom is unnamed, i.e. the columns of the relation have no name-restrictions. Given a rule $r$, we order the atoms from the body of the rule as $A_1, \ldots, A_n$. We will consider each atom in turn and use the body relation $B$ as an accumulator. The equations describing the rule-evaluation process is as follows (the notation is explained below)
\begin{align*}
B^{0} &\leftarrow \top \\
B^{i + 1} &\leftarrow B^{i} \bowtie \sigma_{Term(A_i)}\;(A_i), \; i = 0, \ldots, n - 1\\
H & \leftarrow \Pi_{Term(H)}(B_n)
\end{align*}
\noindent
The body relation is initially assigned to $\top$ denoting an unknown relation: $\top \bowtie R = R \bowtie \top = R$. A selection is then done for the atom. The selection operator is denoted $\sigma$ as is usual in relational algebra (CITE). The selection result is \textit{joined}(CITE) with the current body relation $B^{i}$ to form the next body relation $B^{i + 1}$. The join-operator is denoted $\bowtie$. The $Term$ function gives the terms in the given atom. The project function $\Pi$ projects the corresponding columns into a new relation. In our example (assuming that $r_1$ has been evaluated) we get:
\begin{align*}
B^{1} &\leftarrow \sigma_{x, y}\;(Order) = \{(1, 2), (2, 3)\}_{x, y}\\
B^{2} &\leftarrow B^{1} \bowtie (\sigma_{y, z}\;(Order) = \{(1, 2), (2, 3)\}_{y, z})\\
      &= \{(1, 2), (2, 3)\}_{x, y} \bowtie \{(1, 2), (2, 3)\}_{y, z}\\
      &= \{(1,2,3)\}_{x,y,z}
\end{align*}
\noindent
Finally we project the final result of the body relation into the head of the rule:
\begin{align*}
Order & \leftarrow \Pi_{x,z}(\{(1,2,3)\}_{x,y,z}) = \{(1,3)\}
\end{align*}
\noindent
The process is iterated until a fix-point is found, i.e. until no new tuples can be derived from the set of rules. In our example, iterating $[r_2]$ again gives no new tuples and so the $Order$ relation has been computed as: $\{(1,2), (1,3), (2,3) \}$.

\subsection{Mutual Dependencies and Predicate Ordering}